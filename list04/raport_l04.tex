\documentclass[11pt]{article}
\usepackage[T1]{fontenc}
\usepackage[left=12mm,
right=12mm,top=0.5in,
bottom=1.5in]{geometry}
\usepackage{mathtools}
\usepackage{cancel}
\DeclareUnicodeCharacter{2212}{-}
\begin{document}
\title{Zadanie 7 Lista 4}
\author{Piotr Popis, 245162}
\maketitle
\centering

\begin{flushleft}
\section{Zadanie 1}
\subsection{Opis problemu}
Celem zadania jest implementacja funkcji obliczającej ilorazy różnicowe. Danymi wejściowymi są:
\begin{center}
x- wektor długości n+ 1 zawierający węzły $x_0, . . . , x_n$  $x[1]=x_0,...,x[n+1]=x_n$\\
f– wektor długości n+1 zawierający wartości interpolowanejfunkcji w węzłach $f(x_0), . . . , f(x_n)$\\
\begin{flushleft}
Wyniki:
\end{flushleft}
fx- wektor długości $n+1$ zawierający obliczone ilorazy różnicowe\\ $fx[1]=f[x_0]$,\\$fx[2]=f[x_0, x_1],...,fx[n]=f[x_0, . . . , x_{n−1}],fx[n+1]=f[x_0, . . . , x_n].$
\begin{flushleft}
Addytywnym utrudnieniem jest restrykcja użycia tablicy dwuwymiarowej, czyli macierzy.
\end{flushleft}
\end{center}
\subsection{Rozwiązanie}
Ilorazem różnicowym n- tego rzędu funkcji $f:X\longrightarrow Y$ w punktach $x_0, ..., x_n \epsilon X$ nazywamy funkcję:\\
\begin{center}
$f(x_0,...,x_n) := \Sigma^n_i\dfrac{f(x_i)}{\Pi^n_j(x_i-x_j)}$
\end{center}
W celu realizacji zadania, czyli uniknięcia wykorzystania macierzy skorzystamy z zależności rekurencyjnej:\\
1.$i=0$\\
\quad$f[x_0]=f(x_0)$\\
2.$i=1$\\
\quad$f[x_0,x_1]=\dfrac{f(x_1)-f(x_0)}{x_1-x_0}$\\
3.$i=n$\\
\quad$f[x_0,...,x_n]=
\dfrac{f(x_1,...,x_n)-f(x_0,...,x_{n-1})}{x_n-x_0}$\\
Znając węzły $x_n$ i wartości funkcji f($x_n$), można utworzyć dwuwymiarową tablicę ilorazów różnicowych. Jednak algorytm można zoopytmalizować, ponieważ wystarczy użyć tablicy jednowymiarowej $w$ do zapamiętywania dwóch poprzednich wartości(tablicę aktualizujemy od dołu do góry i od lewej do prawej). Pozostałe wartości tylko i wyłącznie spowalniają nasz algorytm. Początkowymi wartości są $w_i$ są odpowiadające im f[$x_i$]. W kolejnych krokach aktualizowane jest jedno miejsce mniej.
\newpage
\section{Zadanie 2}
\subsection{Opis problemu}
Napisać funkcję obliczającą wartość wielomianu interpolacyjnego stopnia n w postaci Newtona $N_n(x)$ w punkcie x=t za pomocą algorytmu Hornera w czasie $\theta(n)$. Dane wejściowe:\\
\begin{center}
x– wektor długości n+ 1 zawierający węzły $x_0, ...,x_n$\\$x[1]=x0,...,x[n+1]=x_n$\\
fx– wektor długości n+ 1 zawierający ilorazy  różnicowe \\$fx[1]=f[x_0]$,\\
$fx[2]=f[x_0, x_1],...,fx[n]=f[x_0, . . . , x_{n−1}],fx[n+1]=f[x_0, . . . , x_n]$
\end{center}
\begin{flushleft}
Wyniki:
\begin{center}
a – wektor długości n+ 1 zawierający obliczone współczynniki postaci naturalnej\\
$a[1]=a_0$,\\
$a[2]=a_1,...,a[n]
=a_{n−1},a[n+1]=a_n$.\\
\end{center}
\end{flushleft}
\subsection{Rozwiązanie}
W celu wyznaczenia wartości wielomianu interpolacyjnego stopnia n w postaci Newton'a $N_n(x)$ w punkcie $x=t$ zaimplementowano uogólniony schemat Hornera.\\

\end{flushleft}
\end{document}