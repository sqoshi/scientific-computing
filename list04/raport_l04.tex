\documentclass[11pt]{article}
\usepackage[T1]{fontenc}
\usepackage[left=12mm,right=12mm,top=0.5in,bottom=0.5in]{geometry}
\usepackage{mathtools}
\usepackage{cancel}
	
\begin{document}
\title{Zadanie 7 Lista 4}
\author{Piotr Popis, 245162}
\maketitle
\centering

\begin{flushleft}
\section{Treść}
Czy język tych słów nad alfabetem \{1,2,3,4\}, które mają tyle samo symboli 1 co 2 i tyle samo symboli 3 co 4 jest bezkontekstowy?
\end{flushleft}

\section{Rozwiązanie}
Załóżmy, że język L jest bezkontekstowy.
Wtedy skorzystamy z lematu o pompowaniu:\\
Istnieje stała $n$ taka, że jeśli $z$ $\epsilon$ $ L$ $\wedge$ $| z |$ $\geq$ $n$  oraz \\


\end{document}