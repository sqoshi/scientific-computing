\documentclass[11pt]{article}
\usepackage[T1]{fontenc}
\usepackage[left=12mm,
right=12mm,top=1.0in,
bottom=1.5in]{geometry}
\usepackage{amsmath}
\usepackage{mathtools}
\usepackage{cancel}
\usepackage{graphicx}
\usepackage{grffile}
\graphicspath{{/home/piotr/Documents/scientific-computing/list04/exercise5/plots/}{/home/piotr/Documents/scientific-computing/list04/exercise6/plots/}}
\DeclareUnicodeCharacter{2212}{-}
\begin{document}
\title{{Obliczenia Naukowe}}
\author{Laboratorium Lista Nr 5\\Piotr Popis\\ 245162}
\date{6 grudzień 2019}
\maketitle
\centering

\begin{flushleft}
\section{Zadanie 1}
\subsection{Streszczenie problemu}
Problemem jest rozwiązanie równania liniowego $Ax=b$,gdzie $A \epsilon R^{nxn}$ jest podaną macierzą, a $b \epsilon R^n$ zadanym wektorem prawych stron( przy założeniu, iż $n \geq 4$ ).
Dodatkowo macierz A jest macierzą rzadką- taką, która ma dużo elementów zerowych oraz blokową.
\[
A=\begin{bmatrix}
      A_1 & C_1 & 0 & ... & 0 \\
    B_2 & A_2 & C_2 & .. & 0 \\
    \vdots & \ddots & \ddots & \ddots & \vdots \\
     0 & ... & B_{v-1} & A_{v-1} & C_{v-1} \\
      0 & ... & 0 & B_v & A_v \\
  \end{bmatrix}
\]
, gdzie $v=\dfrac{n}{l}$ przy założeniu iż l zawsze dzieli n( n jest podzielne przez l) oraz $l\geq2$. l jest rozmiarem wszystkich kwadratowych macierzy wewnętrznych - bloków: $A_k, B_k, C_k$. Mianowicie: $$A_k \epsilon R^{lxl}, k = 1,...,v  ,$$ A jest macierzą gęstą, \\0 jest kwadratową macierzą zerową stopnia l,\\Natomiast macierz $$B_k \epsilon R^{lxl}, k = 2,...,v  ,$$ $B_k$ ma tylko \underline{dwie ostatnie kolumny niezerowe} i jest postaci:   
\[
B_k=\begin{bmatrix}
      0 & ... & 0 & b^k_{1l-1} & b^k_{1l} \\
      0 & ... & 0 & b^k_{2l-1} & b^k_{2l} \\
      \vdots &     & \vdots & \vdots & \vdots \\
      0 & ... & 0 & b^k_{ll-1} & b^k_{ll} \\
  \end{bmatrix}
\]
Ostani z bloków $$C_k \epsilon R^{lxl}, k = 1,...,v-1  ,$$ $C_k$ jest macierzą diagonalną i jest postaci:
\[
C_k=\begin{bmatrix}
      c^k_1 & 0 & 0 & ... & 0 \\
    0 & c^k_2 & 0 & .. & 0 \\
    \vdots & \ddots & \ddots & \ddots & \vdots \\
     0 & ... & 0 & c^k_{l-1} & 0 \\
      0 & ... & 0 & 0 & c^k_l \\
  \end{bmatrix}
\]
W przypadku naszego problemu n jest ogromne co wiąże się z ograniczeniem - nie używaniem tablicy dwuwymiarowej, która jest czasowo i pamięciowo niewydajna dla dużych n. Prawdopodobnie należy skorzystać z pakietu SparseArrays, która zawiera specjalną strukturę efektywnie pamiętająca specyficznie macierze, tj rzadkość lub regularność występowania elementów zerowych i niezerowych. Istniejące algorytmy do rozwiązywania takich problemów trzeba po prostu zmodyfikować do użycia tej spejcalnej struktury. Jeśli l jest stałe Algorytmy da się zoptymalizować czasowo z $\theta(n^3)$ do  $
\theta(n)$

\subsection{Opis implementacji wraz z analizą złożoności algorytmu}
\subsection{Wyniki eksperymentów porównujących zaimplementowane algorytmy dla danych testowych( tabele, wykresy) oraz interpretacja}
\subsection{Wnioski}
\end{flushleft}
\end{document}